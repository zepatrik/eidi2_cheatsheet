\documentclass[a4paper,7pt,landscape]{article}
\usepackage[T1]{fontenc}
\usepackage[utf8]{inputenc}
\usepackage{lmodern}
\usepackage[german]{babel}
\usepackage{parskip}
\usepackage{tabularx}
\usepackage{amsmath}
\usepackage{amssymb}
\usepackage{multicol}
\usepackage[left=1cm,right=1cm,top=1cm,bottom=1cm]{geometry}
\usepackage{hyperref}
\usepackage{stmaryrd}
\usepackage{tikz}

\usetikzlibrary{calc,positioning,shapes}

\newcommand{\semT}[1]{\llbracket{}\texttt{#1}\rrbracket{}}
\newcommand{\WP}[1]{\textsf{WP}\semT{#1}}

\title{EIDI 2 Cheatsheet}
\author{github contributors: \url{https://github.com/zepatrik/eidi2_cheatsheet} \\
licence CC BY-SA
}

\begin{document}
  \maketitle
  \newpage
  
  \begin{multicols*}{3}
    \section{Logik}
    \begin{align*}
      \neg (A \lor B) &\equiv \neg A \land B \\
      A \lor (B \land A) &\equiv A \land (B \lor A) \equiv A \\
      A \implies B &\equiv \neg A \lor B
    \end{align*}
    
    \section{Verifikation}
    \subsection{WP}
    \begin{align*}
      \WP{x = read();}(B) &\equiv \forall x.B \\
      I \Rightarrow \WP{b}(B_0, B_1) &\equiv I \Rightarrow (((\neg b) \Rightarrow B_0) \land (b \Rightarrow B_1))
    \end{align*}
    \subsection{Terminierung}
    \begin{enumerate}
      \item vor jedem Schleifendurchlauf \texttt{r > 0}
      \item \texttt{r} wird bei jedem Durchlauf kleiner
    \end{enumerate}
    
    \begin{tikzpicture}[
    node distance=2em,
    font=\ttfamily,
    rect/.style={rectangle,draw},
    diam/.style={diamond,draw,aspect=2}
    ]
      \node[rect] (1) {r = x + y;};
      \node[above=of 1,yshift=-1em] (0) {};
      \node[diam,below=of 1] (2) {x != y};
      \node[rect,right=of 2] (3) {assert(y > 0);};
      \node[diam,below=of 3] (4) {x < y};
      \node[rect,below left=of 4] (5) {x = x - y;};
      \node[rect,below right=of 4] (6) {y = y - x;};
      \node[rect,below=of 4,yshift=-1.75em] (7) {assert(y > \textcolor{blue}{x + y});};
      \node[rect,below=of 7] (8) {y = \textcolor{blue}{x + y};};
      \node[left=of 2,xshift=1em] (exit) {};
      
      \draw[->] (0) edge (1) (1) edge (2) (3) edge (4) (7) edge (8);
      \path[draw,->] (4) -| node[above] {no} (5) (4) -| node [above] {yes} (6) (2) -- node [above] {yes} (3) (2) -- node[above] {no} (exit);
      \draw[->,to path={|- (\tikztotarget)}] (5) edge (7) (6) edge (7);
      \path[draw,->] (8) -| (2);
    \end{tikzpicture}
    
  \end{multicols*}
\end{document}
